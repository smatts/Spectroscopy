\documentclass[12pt,a4paper]{article}
\usepackage[latin1]{inputenc}
\usepackage{amsmath}
\usepackage{amsfonts}
\usepackage{amssymb}

\title{Shot noise limit (Data: 040719) }
\setlength{\parindent}{0pt}

\begin{document}

\maketitle

\section{Calibration of the detection path}

The detection path from 'power at the bar code reader' to 'lock-in display' needs to be calibrated. Needed: 'noise source' with know characteristic .

\begin{itemize}
  \item 594 nm HeNe through AOM (400 kHz rectangular on/off modulation) on pump-beam path to detector
  \item Power at detector: 540 nW (at 594 nm), measured cw averaged with Newport Diode-Powermeter
  \item 12 MHz low-pass filter installed, lock-in settings as for measurement (3~ms, 24~dB/oct)
  \item Lock-in reads 4.45~mV$_{\text{RMS}}$, scope reads 4.60~mV$_{\text{RMS}}$ and 12.2~mV$_{\text{pk-pk}}$
\end{itemize}

Which of these values is the relevant one? The light intensity goes 0 -- 1 -- 0, but the voltage after the AC coupled detector goes -0.5 -- +0.5 -- -0.5. This gives different RMS-values ($\sqrt{0.5}$ vs 0.5), so I use the peak-peak values.

This gives a sensitivity of 
\[
     \frac{12.2 \, mV}{1080 \, nW} = 11.2\, kV/W \]
The datasheet gives a value of 25  kV/W, but the adjustable gain was probably set lower than 30.

The spectral response of the detector C5331-11 differs a bit between  594~nm (14.1~arb.~units) and  520~nm (12.9~arb.~units). This gives a final value for sensitivity at 520 nm of \textbf{10.2 kV/W}. 



\section{Noise measurement}

The interferometer was set to 4 times dark fringe ($\approx 4 \, \mu W$) and the standard deviation of 100 measurements every 10 ms was calculated to 130~nV. This gives
\[
  \sigma_{\text{exp}} = 130 \, nV \; / \; 10.2 \; kV/W = 13 \; pW \]
 

\section{Noise estimation}

For the power at the detector used above ($ 4 \, \mu W$) the expected noise is calculated.
From e.g. H.-A. Bachor \emph{A guide to experiments in quantum optics} 
\[ 
  \sigma = \sqrt{ 2 \; \Delta \nu \; P \frac{h \, c}{\lambda}} \]
where $\Delta \nu$ is the bandwidth of the measurement, $P$ the power at the detector and $\sigma$ the noise (std.dev.) in Watts.

The filter response function of the lock-in is not rectangular, but an equivalent noise bandwidth (ENBW) can be calculated (see SR844 manual). It depends on integration time $T$ (3 ms) and roll off (24 dB/oct):
\[
  \Delta \nu_{\text{eq}} = \frac{5}{64 \, T} \qquad \text{for 24 dB/oct slope} \]
 so $\Delta \nu = 26$~Hz.
  Everything together gives
 \[ 
 \sigma_{\text{th}} = 9 \; pW \]

We see about 1.4 times more noise than shot noise! This corresponds to an APD excess  noise factor of $1.4^2 = 2$, a reasonable value.
 
\end{document}