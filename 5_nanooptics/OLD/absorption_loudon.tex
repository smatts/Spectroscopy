%\documentclass[a4paper,10pt]{scrartcl}          %twoside
\documentclass[a4paper,10pt]{article}
\usepackage[utf8x]{inputenc}
\usepackage[left=3cm,right=2cm,top=2.5cm,bottom=2.5cm,a4paper,includehead]{
geometry}
\usepackage{amsmath}
%opening

\title{Integral absorption of an emitter}
%\date{}
\author{}
\pagestyle{empty}


\begin{document}


\maketitle

\pagestyle{empty}
\thispagestyle{empty}

This follows R. Loudon, The quantum theory of light.

The attenuation coefficient $K(\omega) = 2\omega \; \kappa(\omega) /c$
describes the intensity after propagation over a distance $z$ (Loudon eq. 1.8.3)
\[
 \frac{I(z)}{I(0)} = \exp \left( - K (\omega) \; z \right)
\]
It is connected to the Einstein B coefficient (Loudon eq. 1.8.17)
\[
 \int_0^{\infty} d\omega \; \frac{K(\omega) \; c}{\hbar \; \omega} =
\frac{N}{V} \; B
\]
where $N$ emitters are present in the volume $V$ (see also Petzke
script eq. 4.104).

Together we get
\[
  \frac{I(z)}{I(0)} = 1 - \frac{\Delta T}{T} = \exp \left( - K (\omega) \; z
\right)
\]
resulting in
\[
z K(\omega) = -   \ln \left(  1 - \frac{\Delta T}{T} \right)
\]
Assuming a focal volume   $V = A \cdot z$ we get
\[
 B = \frac{z A}{N}   \int_0^{\infty} d\omega \; \frac{K(\omega) \; c}{\hbar \;
\omega} 
 = \frac{ A}{N} \int_0^{\infty} d\omega \; \frac{-\ln \left(  1 - \frac{\Delta
T}{T} \right)  \; c}{\hbar \; \omega}
%
\approx 
    \frac{ A}{N} \int_0^{\infty} d\omega \; \frac{  \frac{\Delta
T}{T}    \; c}{\hbar \; \omega}
\]




The Einstein B coefficient is related to the dipole moment $\mu = e \cdot D$ via
 (Loudon eq. 2.3.20)
\[
 B = \frac{\pi \; \mu^2}{3 \; \epsilon_0 \; \hbar^2}
\]
where the factor $1/3$ comes from averaging over all orientations.

Together we get 
\[
 N \; \mu^2 =   \frac{3 \; \epsilon_0 \; \hbar^2}{\pi} \; A \;
\int_0^{\infty} d\omega \; \frac{  \frac{\Delta
T}{T}    \; c}{\hbar \; \omega}
\]


The dataset \verb|fieldonqwires11_wire1_longtime| has a peak transient
transmission value of $\Delta T /T = 3 \cdot 10^{-6}$. The probe spot has a FWHM
of 400 nm, resulting in a probe area $A = 1.13 \cdot 400^2 = 1.8 \cdot
10^5$~nm$^2$. This gives a peak $\sigma_0 = 0.55 $~nm$^2$. The integrate gives
the value
\[
 N \; \mu^2 = 1250 \text{ D}^2
\]

An emitter at $\lambda = 600$~nm with a radiative lifetime of $\tau = 10$~ns
in vacuum has an oscillator strength $f=1.6$ (Karrai/Warburton 2003, eq. 20).
It has a dipole moment of 8.3~D (K/W 2003, eq 12). We would need about 18 of
such dipoles to make the signal, exactly the value we had yesterday with the
other estimate.


\end{document}
