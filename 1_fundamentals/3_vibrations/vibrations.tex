\renewcommand{\lastmod}{May 1, 2020}


\chapter{Molecular Vibrations}



\section{Tasks}

\begin{itemize}
\item Get  the raw data of the absorption and emission spectra of the dye BODIPY 650/665 from \href{https://www.thermofisher.com/de/de/home/life-science/cell-analysis/labeling-chemistry/fluorescence-spectraviewer.html?SID=srch-svtool&UID=10001moh}{thermofischer.com}.
 Convince yourself that the mirror-law holds.




\item Make an as accurate as possible sketch of the potentials of ground and excited state of BODIPY 650/665, assuming that only one vibrational mode contributes.

\end{itemize}


%\begin{marginfigure}
%   \includestandalone[width=\textwidth]{\currfiledir fig_bodipy}
%  \caption{Absorption and emission spectrum of  the BODIPY dye.}
%\end{marginfigure}
%
%



\section{Franck-Condon and Huang-Rhys}

A molecule with an electronic ground state $g$ and electronic excited state $e$ can undergo periodic oscillations of the nuclei positions along a coordinate $r$. We assume\footcite{Kuzmany} that the potential of these oscillations is harmonic, i.e.
\begin{eqnarray}
 U_g(r) &=& \frac{1}{2} \, k\, r^2 = \frac{1}{2} \, m \, \omega^2 \, r^2 \\
  U_e(r) &=&  U_g(r) + E_{eg} - A \, r = E_{eg}  - A \, r + \frac{1}{2} \, m \, \omega^2 \, r^2 
 \end{eqnarray}
where $E_{eg}$ is the electronic contribution to the energy difference. We assumed that both potentials have the same shape, i.e., the same vibrational frequency. The term $A \, r$ couples the electronic state and the nuclear oscillator. It shifts the excited state potential along the $r$ coordinate. While the ground state potential has its minimum at $r=0$, the minimum for the excited state is at
%
%
\begin{marginfigure}
   \includestandalone[width=50mm]{\currfiledir fig_parabola}
\caption{The coupling term $-A r$ in the potential of the excited state $e$ shifts the minimum of the parabola to larger values of $r$ and lower values of the potential. }
\end{marginfigure}
%
%
\begin{equation}
 A = m \, \omega^2 \,  r	 \quad \text{i.e.} \quad r = \frac{A}{m \omega^2}
\end{equation}
The energies of the quantum mechanical eigenstates are 
\begin{eqnarray}
  E_{g, n} &=&  (n + 1/2) \, \hbar \omega  \\
  E_{e, m} &=&  (m + 1/2) \, \hbar \omega  +  E_{eg} - \frac{A^2}{2 m \omega^2} =
   (m + 1/2 - S^2) \, \hbar \omega  +  E_{eg} 
\end{eqnarray}
We introduce the Huang-Rhys factor $S$ as dimensionless coupling constant
\begin{equation}
 S = \frac{A}{\hbar \omega} \sqrt{\frac{\hbar}{2 m \omega}} \quad .
\end{equation}
The eigenfunctions $\chi_n$ of the nuclear vibrations are Hermite polynomials. The Franck-Condon factor describes the overlap integral of the vibrational wavefunction of ground and excited state. As the electronic transition is fast compared to nuclear motion, the nuclear coordinate cannot change during the transition (Born-Oppenheimer approximation), and both ground and excited state need a non-vanishing probability to be at the same coordinate $r$. When one of the states is in a vibrational ground state, i.e., $n$ or $m$ equals zero, the Franck-Condon factor takes the form\sidenote{This notation is sloppy in the sense that the bra wavefunction is an electronic excited state, the ket function an electronic ground state!}
\begin{equation}
 | \braket{ \chi_0 | \chi_m } | ^2  =  | \braket{ \chi_m | \chi_0 } | ^2 = \frac{S^m \exp(-S)}{m!}
\end{equation}
which is a Poisson distribution of mean value $S$.  The strongest transition is thus the transition into $m \approx S$, which for large coupling between electronic and nuclear system, i.e. large $S$, will deviate from the 0--0 transition.

\begin{figure}
   \includestandalone[width=106mm]{\currfiledir fig_poisson}
  \caption{Poisson distributions}
\end{figure}

The Debye-Waller factor $D$ gives the ratio of the coherently scattered wave to all scattering processes. For molecules, this corresponds to the amplitude of the 0--0 line to the integral over the whole band. As the sum over all Franck-Condon factors to the same final state is one, we get
\begin{equation}
 D =  | \braket{ \chi_0 | \chi_0 } | ^2 = \exp(-S)
\end{equation}


\section{Stokes shift and mirror rule}


When we keep the assumptions of the last section, that the vibrational frequency of ground and excited state is the same, both potentials are harmonic, and of course the Born-Oppenheimer approximation hold, then absorption and emission spectrum are closely related. The 0--0 transition at energy $E_{00}$ from the vibrational ground state of the electronic ground state to the vibrational ground state of the electronic excited state appears both in absorption and emission. As the thermal energy $kT$ is in most cases small compared to the vibrational energy $\hbar \omega$, almost all molecules are in the vibrational ground state. Absorption then only occurs at energies larger than $E_{00}$ into higher vibrational state of the electronic excited state. These energies are
\begin{equation}
  E_{abs, n} = E_{00} + n \, \hbar \omega
\end{equation}
Fluorescence emission also occurs out of a vibrational ground state, but due to different reasons than absorption. In molecules, vibrational relaxation  (some ps) is much faster\sidenote{'Faster' means here that the rates are larger. The event itself can be assumed to be instantaneous. } than fluorescence emission (some ns). The emission occurs thus into different vibrational levels of the electronic ground state at
\begin{equation}
  E_{em, n} = E_{00} - n \, \hbar \omega
\end{equation}
The spectral position of the absorption and emission peaks are thus mirrored\footcite[chapter 1.3.2 and 1.3.3]{Lakowicz2010} around the 0--0 transition  energy $E_{00}$.

Not only the spectral positions, but also the amplitude of the peaks in absorption spectrum $\epsilon(\omega)$ and fluorescence spectrum $F(\omega)$ are related. The reason is that the Einstein $A$ and $B$ coefficients are related, or that there is only one transition dipole moment $\mu$ which governs both absorption and emission. The only caveat is the relation between the transition dipole moment and the spectra\footcite[Chapter 5.2]{Parson}
\begin{eqnarray}
   \epsilon(\omega  =  \omega_{g,m \rightarrow e,n} )  & \propto & \omega_{g,m \rightarrow e,n}  \,  | \braket{\chi_n |  \chi_m} |^2 
\, B_{eg} \\
   F(\omega =  \omega_{e,n \rightarrow g,m} ) & \propto & \omega_{e,n \rightarrow g,m}^3 \,  | \braket{\chi_m |  \chi_n} |^2 
\, B_{ge}
\end{eqnarray}
In each case, one factor of $\omega$ appears due to the photon energy $\hbar \omega$, as the right-hand side considers single absorption or emission events,  but the right-hand side uses spectra in terms of power per spectral interval, not photons. The fluorescence spectrum gets an additional factor of $\omega^2$ due to the optical mode density in 3D space, as it enters the black-body spectrum and the relation between the Einstein $A$ and $B$ coefficients.
Taking everything together, one should therefor compare $\epsilon / \omega$ and $F / \omega^3$.







\printbibliography[segment=\therefsegment,heading=subbibliography]
