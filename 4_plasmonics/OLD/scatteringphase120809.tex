\documentclass[a4paper,10pt]{article}
\usepackage[utf8x]{inputenc}
\usepackage{amsmath}

%opening
\title{On the scattering phase}
\author{}

\begin{document}

\maketitle

Assume a plane wave traveling in postive z direction, polarized in x direction
\[
 E = E_0 e^{ i (kz - \omega t) }
\]

The wave is sctterd on an object, resulting in a scatterd wave  (BH eq. 5.17)
\[
 E_s = \frac{e^{ik (r - z)}}{-i k r} {\bf X } E
\]


We start from a diven dipole at $z=0$
\[
 p = \epsilon_m \alpha E_0
\]
with
\[
 \alpha = 4 \pi a^3 \frac{\epsilon_1 - \epsilon_m}{\epsilon_1 + 2 \epsilon_m}
\]

The full field of an oscillating dipole is
\[
E_s = \alpha \frac{e^{i k r}}{4 \pi } \left(  
% FF
 \frac{-k^2}{r}  \; e_r \times  (e_r \times e_x)
% MF + NF
 + \left[ 3 e_r (e_r \cdot e_x) - e_x \right]
\left( \frac{1}{r^3} - \frac{i k}{r^2} \right)
 \right) \; E_0
\]

For points on the z axis ($e_r = e_z$) this results in 
\[
E_s = e_x \alpha \frac{e^{i k r}}{4 \pi } \left(  
% FF
 \frac{ k^2}{r} 
% MF + NF
 -
 \frac{1}{r^3} + \frac{i k}{r^2}  
 \right) \; E_0
\]

This means the dipole scatterd field is 180 deg out of phase between near- and
far-field. The dipole far-field is proportional to $\alpha E$.




The vector scattering amplitude $\bf{X}$  (BH p. 70) can be calculated for the
dipole limit for points on the z axis ($e_r = e_z$)
\[
 {\bf X} = e_x  \frac{\alpha}{4 \pi } \left(  
% FF
  -i k^3  
% MF + NF
 +
 \frac{i k }{r^2} + \frac{ k^2}{r}  
 \right)  
\]

Mie theory allows us to make the transition from the dipole limit to the mirror
limit, increasing the size of the sphere. We end with a (spherical) mirror.

For a (plane) mirror, the reflected field is
\[
 E_r = r E  \quad \text{with} \quad r = \frac{1 - n}{ 1 + n}
\]
assuming that the environment is vacuum. For a plane mirror, the field does not
depend on distance to the mirror. We can not excite, e.g., plasmons in the
mirror.

An ideal mirror does not absorb. The absorption cross section is zero,
scattering equals extinction and approaches the geometrical cross section. This
means, as
\[
 C_{ext} = \frac{4 \pi}{k^2} \Re \left( {\bf X}_{forward} \right)
\]
that $X$ is purely real, i.e., the forward scattered field (through the mirror)
is purely imaginary. As this is on axis, we have the constructive interference
of the 0. order diffraction even behind a black disc. It decays fast when going
away from the optical axis (see BH p.110).


For a dipole on resonance, both the forward and backward scatterd fields are
$+i$ relative to the driving field. There is no absorption detectable directly
on the optical axis, rather an increase in intensity. When tuning the
wavelength over theresoance, one gets a dispersive lineshape. A little bit away
from resonance we still have field strength, but also some real-valued
components. (See Ilja Gerhardt dissertation figure 2.7, p.18). When we increase
the detector size, only then we detrect absorption.

Or we use a Gaussian focus. The trasmitted beam makes a -90 deg phase shift into
the far-field (for the sign see Born/Wolf p.448, or Milonni). This is then 180
deg out-of-phase with the $+i$ phase of the dipole on resonance. Then we can
detect absorption also with a point detector on axis (agrument from Iljas
thesis).



\end{document}
