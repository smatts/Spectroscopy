

%\renewcommand{\lastmod}{April 29, 2020}

\chapter{Plasmon hybridization}

\section{Tasks}

\begin{itemize}
\item Derive analytical solutions for the  amplitude and position of the two hybridized peaks in the absorption spectrum of two neighbouring particles in the Rayleigh limit. 

\item Use the hybridization model to explain the antenna effect that is used to amplify the transient transmission signal of a small gold particle in Schumacher et al. XXX.

\end{itemize}


%--------------

\section{Overview}


The concept of plamson hybridzation helps to get a more intuitive understanding of the absorption spectra of arrangements of plasmonic nanoparticles. 



\section{Rayleigh scattering of small spheres}


Let us start by goiing back to  Rayleigh scattering of nanoparticles, as we diiscussed already in chapter XXX. A sphere of radius $R$ and dielectric constant $\epsilon_{in}$ is embedded in a medium of dielectric constant $\epsilon_{out}$. We assume that the radius $R$ is much smaller than the wavelength $\lambda$ of the electromagnetic light field. This means that the phase is constant across the sphere and that we can employ the quasi-static approximation. One solves the Laplace equation taking  boundary conditions and symmetry into account.\footcite{Jackson-ED}\footcite[excercise 2.4.2]{Nolting-ED}\footcite[chapter 5.2]{BH-book}
The sphere responds to the light field with a polarization of
\begin{equation}
 \mathbf{p}(t) = \epsilon_0 \,  \epsilon_{out} \, \alpha \, \mathbf{E}(t)
\end{equation}
with the polarizability
\begin{equation}
 \alpha = 4 \pi  \; R^3 \; \frac{\epsilon_{in} - \epsilon_{out}}{\epsilon_{in} + 2 \epsilon_{out}} \quad .
\end{equation}
We find a resonance when $\epsilon_{in}(\omega) + 2 \epsilon_{out}(\omega) = 0$, which requires one dielectric function to be negative, as it is the case in metals. Small metal particles show thus exceptional strong interaction with light in a certain spectral range.




As the electric field oscillates $E(t) = E_0 \, e^{-i \omega t}$, also the polarization $p$ oscillates and radiates a secondary, scattered electromagnetic field 
\begin{equation}
  \mathbf{E}_S = \frac{ e^{i \, k  r} }{4\pi\epsilon_0 \, \epsilon_{out}}  \frac{1}{r^3}\left\{
      (k r )^2 \left( \hat{\mathbf{r}} \times \mathbf{p} \right) \times \hat{\mathbf{r}} +
      \left( 1 -  i k r \right)
        \left( 3\hat{\mathbf{r}} \left[\hat{\mathbf{r}} \cdot \mathbf{p}\right] - \mathbf{p} \right)
    \right\} \quad ,
\end{equation}
where $k = 2 \pi / \lambda$ is the length of the wave vector in the medium. The power that is absorbed by the dipole\footcite[Chapter 8]{Novotny-Hecht2012} is
\begin{equation}
 P_{abs} = \frac{\omega}{c} \, \Im \left( \mathbf{p} \, \mathbf{E}^\star \right)  \quad ,
\end{equation}
so that we get the absorption cross section
\begin{equation}
 \sigma_{abs} = k \, \Im ( \alpha ) =  4 \pi \, k \; R^3 \; \Im \left( \frac{\epsilon_{in} - \epsilon_{out}}{\epsilon_{in} + 2 \epsilon_{out}} \right) \quad .
 \label{eq:hybrid_sigma_abs}
\end{equation}
We are in the Rayleigh  limit of a very small particle so that we can neglect the scattered power. in this way, the absorption cross section
$ \sigma_{abs} $ equals the extinction  cross section $ \sigma_{ext} $ 


We  assume that the surrounding  medium  is a transparent dielectric, i.e., 
$\epsilon_{out}$ is real-valued. The material of the nanosphere should be described by the Drude model of metals. This is often the case when one is far enough away from inter-band transitions that lead to the color of metals, i.e., when one is far enough in the infrared. The dielectric function then reads
\begin{equation}
 \epsilon_{in} (\omega) = \epsilon_{\infty} - \frac{\omega_P}{ \omega \left(\omega \;
+ \; i\, \gamma \right) } \quad ,
\end{equation}
where $\epsilon_{\infty} $ is the  high-frequency limit, $\omega_P$ the plasma frequency and $\gamma = 1 / \tau_\text{coll} $ the damping parameter of the plasma oscillation. The plasma frequency depends on the effective electron mass $m^\star$ and number density $n$ as
\begin{equation}
\omega_P = \frac{n \, e^2}{m^\star \epsilon_0} \quad.
\end{equation}


The polarizability $\alpha$ has a resonance when its denominator equals zero, i.e., at $\epsilon_{in} (\omega_{res}) = -2 \epsilon_{out}$. For a Drude metal with low damping this happens at
\begin{equation}
\omega_{res} = \frac{\omega_P}{\sqrt{2 \epsilon_{out} + \epsilon_\infty}}
\end{equation}
The resonance wavelength in the absorption spectrum thus depends on 
the plasma frequency of the metal and  the dielectric function of the environment. 


 %----------
\section{Plasmon hybridization}

Plasmon hybridization is another incarnation of a coupled oscillator, i.e., two pendula coupled by a spring. The coupled system has new eigen-functions and eigen-energies that can be derived from the old eigen-functions and the strength of the coupling. The term is borrowed from the hybridization of atom orbitals, for example in carbon atoms forming  the famous sp$^3$ orbitals.

Lets investigate the optical properties of two small Rayleigh particles which are brought close to each other.  The optical response of each particle is described by  an dipole $ \mathbf{p}_i(t)$, where $i = 1,2$.
 Each dipole experiences the incident field
$\mathbf{E}^{\text{inc}}(\mathbf{r}_i)$ and the field scattered from the other dipole.
The sum of these two fields multiplied by the dipoles polarizability $\alpha_i$
has to give in a self-consistent way the dipole moment (see, for example, \cite{Myroshnychenko08})
%
\begin{equation} \label{eq:equationsystem}
     \mathbf{p}_1 = \epsilon_0 \,  \epsilon_{out} \,  \alpha_1 \left[ \mathbf{E}^{\text{inc}} (\mathbf{r}_1) +
\mathbf{E}^{\text{scat}}_2(\mathbf{r}_1) \right] \quad ,
\end{equation}
%
and vice versa. The scattered electrical near field $ \mathbf{E}^{\text{scat}}$ of the dipole $i$ at position of the dipole $j$ is given by eq XXX above. As we aim  for a large influence of this scattered field, we will need short distances between the dipoles and thus can focus on the near-field contribution of the scattered field
\begin{equation}
  \mathbf{E}^{\text{scat, nf}}_i(\mathbf{r}_j) = \frac{ 1 }{4\pi\epsilon_0 \, \epsilon_{out}}  \frac{1}{d^3}
        \left( 3\hat{\mathbf{r}}_{ij} \left[\hat{\mathbf{r}}_{ij} \cdot \mathbf{p}_i \right] - \mathbf{p}_i \right)
  \quad ,
\end{equation}
where $\hat{\mathbf{r}}_{ij}   = \mathbf{r} _j - \mathbf{r} _i$ is a vector of length one pointing from the dipole to the point where
the field is evaluated, and $d$ is the distance between the particles


For simplicity, we assume that both particles have the same dielectric function and are of course embedded in the same medium. As also  in other examples of hybridization, the  total number  of eigen-function has to remain constant. We start with two functions an d thus should get two  solutions  of the equation system \ref{eq:equationsystem}.
Additionally, we  can chose the polarization direction of the incoming electric field  $\mathbf{E}^{\text{inc}}$. Things become simple when we chose it to be either parallel or perpendicular to the connecting axis of the particles. In both cases, the scattered near-field at particle $j$ has the direction of the dipole $i$, which is not the case for other polarization directions.
In total, we obtain the resonance
frequency $\omega_{\text{res}} $ of the coupled two-particle system (see, for example, \cite{Myroshnychenko08} )
%
\begin{equation}  \label{eq:omega_coupled}
 \omega_{\text{res}} = \frac{\omega_P}{\sqrt{2 \epsilon_{out} + \epsilon_{\infty}} }  \; \sqrt{
\frac{1 + g}{ 1 +  \eta \; g}}
\end{equation}
%
with 
%
\begin{equation} \label{eq:omega_coupled_variables}
 \eta = \frac{\epsilon_{\infty} - \epsilon_{out} }{\epsilon_{\infty} + 2 \epsilon_{out}  } 
 \qquad \text{and} \qquad
    g = m \;  \left( \frac{\sqrt{R_1 \; R_2 } } { d }  \right)^3 \quad .
\end{equation}
%
In the case of gold particles in vacuum, the factor $\eta$ takes a value of about $8/11 \approx 0.73$.
For the electric field being parallel to the pair axis, the index $m$ assumes
the value $-2$ for parallel dipoles (head to tail) and $2$ for anti-parallel
dipoles (head-to-head). When the electric field is perpendicular to the
pair-axis, $m$ is $+1$ for the parallel configuration and $-1$ for the
anti-parallel configuration. This is the classical electrodynamics analogon  of H and J aggregates in coupled dye molecules, discussed in chapter XXX.

\begin{marginfigure}

\caption{Level scheme}
\end{marginfigure}

As in the case of molecular aggregates, the combined oscillator strength of both particles is re-distributed over the two new peaks in the absorption spectrum. 
For
the coupled system, the absorption spectrum is given by  
%
\begin{equation} 
 C_{abs} =  \frac{k}{|\mathbf{E}^{\text{inc}}|^2} \; \Im (\mathbf{E}_1 \cdot \mathbf{p}_1 + \mathbf{E}_2 \cdot \mathbf{p}_2) \quad ,
\end{equation}
%
where $\mathbf{E}_j$ is the total electric field at position $j$. The spectrum shows  Lorentzian
peaks at the symmetric (+) and antisymmetric (-) resonances.  When  keeping the coupling factor $g$ in eqs. (\ref{eq:omega_coupled}) and (\ref{eq:omega_coupled_variables})
constant, we find the  amplitudes of
the two peaks  proportional to\sidenote{We assume small splitting so that the wave vector $k$ does not vary much in eq.  \ref{eq:hybrid_sigma_abs}} 
%
\begin{equation} 
 A_\pm = \frac{1}{2} \left( R_1^{3/2} \pm R_2^{3/2} \right)^2 \quad .
\end{equation}
%
The sum of both amplitudes is, as expected, proportional to the total oscillator
strength
%
\begin{equation} 
 A_+ + A_- = R_1^3 + R_2^3 \quad .
\end{equation}
%
For two equal particles ($R_1 = R_2$), the symmetric mode caries twice the
oscillator strength of a single  particle and the antisymmetric mode
is dark. 


\section{Real metals}

In the last section, we assumed a Drude metal for both particles. This allowed us to give analytical expressions for peak positions and width. But of course plasmon hybridization also exists for real metals. In stead of the Drude formula eq. XXX we use measured dielectric functions $\epsilon_{in}$, for example from Johnson and Christy XXX. We assume an incoming polarization direction $\mathbf{E}^\text{inc}$ and wavelength $\lambda$. Then we solve the equation system given by eq XXX (and the same with swapped indices) to obtain the dipole amplitudes and directions $\mathbf{p}_i$. With this we can calculate the scattered field (eq. XXX) and the absorption cross section (eq XXX). To get the full absorption  spectrum we iterate over the wavelength $\lambda$.


The effect of a real metal is additional damping due to interband absorption. For gold this happens at wavelengths below about 520 nm, leading to the color of gold. With $\omega_P = 9 eV$, $\epsilon_\infty = 6$ and vacuum as medium ($\epsilon_{out} = 1$), the plasmon resonance would appear in the Drude model at $\omega_{res} \approx 3.2$ eV or $\lambda = 390$ nm. The interband absorption shifts the resonance position to about 530 nm wavelength, just at the rim of the absorption band. Plasmon hybridization the splits the peak. The lower wavelength / higher frequency peak overlaps more with interband absorption and will be damped out. Splitting of peaks is thus difficult to observe for small gold nanoparticles.



\section{Beyond the Rayleigh approximation}

We used the Rayleigh approximation, i.e. assumed that each particle is much smaller than the wavelength of light. Such small particles have only very small polarizabilities $\alpha$, as these scale as teh volume of the particle. To obtain sizeable effects, one thus uses particles that are a bit larger, i.e.. smaller but not much smaller than the wavelength. 

We discussed the Mie formalism (chapter XXX) as method to model the optical response of spheres of arbitrary size. It should be in principle possible to model two neighbouring spheres using Mie scattering for each sphere, but this get a bit tedious as the scattered field is not homogeneous over the receiving sphere. More general numerical method such as the finite element method (FEM) or discrete dipole approximation (DDA) are better suited.

The effect of plasmon hybridization exists also for larger and also for non-spherical particles. Especially when the distance between the particles is not large anymore to their size, simple models relying on a few dipoles break down.


\section{Ultrafast optical response of metals}



In the pump-probe experiment, a pump pulse modifies the dielectric properties of
the particle. In a first approximation we can assume that the plasma frequency
$\omega_P$ is shifted by the particle expansion to $\omega'_P = \omega_P (1 +
\delta)$ with $\delta \ll 1$. The new resonance position is then
%
\begin{equation} \label{eq:omega_equal}
 \omega_{\text{res'}} = \frac{\omega_P (1 + \delta)}{\sqrt{2 +
\epsilon_{\infty}} }  \quad .
\end{equation}




\section{Pump-induced spectral shifts: 2 particles}



In the pump-probe experiment, we are only interested in the influence of the
small particle's variation on the extinction spectrum. We therefore vary now the
plasma frequency of only one particle by $\delta \ll 1$ and calculate the new
resonance positions. We get
%
\begin{equation} \label{eq:omega_diff}
 \omega_{\text{res'}} = \frac{\omega_P (1 + \delta / 2)}{\sqrt{2 + \epsilon_{\infty}} }
 \; \sqrt{ \frac{1 + g}{ 1 +  \eta \; g}} \quad .
\end{equation}

Plasmon hybridization does not increase the amount by which the resonance is
shifted upon changing the plasma frequency of one sphere only. The shift is reduced by a factor of $2$ when comparing 
 eq. (\ref{eq:omega_diff})  with  eq.
(\ref{eq:omega_equal}). This can be understood
as we modify only part of the system,  in most cases even less than half of the
system's total volume.


However, the shift of the resonance position is only part of the answer to
signal enhancement, as we detect changes in transmission. Our signal is
proportional to the product of resonance shift and peak height of the extinction
resonance. A stronger extinction peak can overcompensate the reduced shift. For
the two-sphere system, the extinction spectrum is given by  eq XXX

Already this twice as strong peak would compensate for the
reduction in peak shift calculated above. However, the antenna would not
enhance the signal. As soon as the second sphere becomes larger ($\zeta > 1$),
the symmetric mode continuously increases in amplitude and the antenna starts to
enhance the signal. In the dipole approximation we find  no upper bound for the
antenna enhancement.




\printbibliography[segment=\therefsegment,heading=subbibliography]
