\renewcommand{\lastmod}{April 15, 2020}


\chapter{Strong coupling of cavity and emitter}




\section{Tasks}

\begin{itemize}
\item Pelton data ?
\end{itemize}



\section{Experiment}

some text


\section{Quantization of the light field}
The Rabi model assumed a classical electrical field in which the particle nature of light is not relevant. But now we need to take the quantized nature of light into account. In quantum mechanics this is often called 'second quantization' and we will briefly have a look at the main results.

The principle idea is very similar to a quantum mechanical harmonic oscillator, i.e. a series of equidistant states that has a bottom boundary. We describe the states by a quantum number $n$, starting from $n=0$. The Hamiltonian reads
\[
\hat{H} \ket{n} = E_n \ket{n} = \left( n + \frac{1}{2} \right) \, \hbar \omega \, \ket{n}
\]
where $\hbar \omega$ is the energy distance between the states.

It is convenient to use ladder operators for the creation ($\hat{a}^\dagger$) and annihilation ($\hat{a}$) of  a quantum of energy, i.e.
\[
 \hat{a}^\dagger \ket{n} = \sqrt{n+1}\, \ket{n+1}  \quad \text{and} \quad
  \hat{a} \ket{n} = \sqrt{n}\, \ket{n-1}
\]
Useful properties are 
\[
 \hat{a} \ket{0} = \ket{0}  \quad \text{and} \quad
  \hat{a}^\dagger  \hat{a} \ket{n} = n \ket{n}
\]

Now this needs to be connected to  classical electrodynamics. We assume a single optical mode in a small optical resonator, similar to a laser cavity. In the dark, i.e. in the state $\ket{0}$, quantum mechanics gives an eigen-energy $E_n = 1/2 \, \hbar \omega$. This is what we require also from classical electrodynamics:\footcite[chap. 7.5]{Fox}
\[
E_0 = 
 \int_\text{cavity} \frac{1}{2} 
 \left( \boldsymbol{H} \cdot  \boldsymbol{B} + \boldsymbol{E} \cdot  \boldsymbol{D} \right) \, dr = 
  \int_\text{cavity}  \epsilon_0 \boldsymbol{E}^2 \, dr = \frac{1}{2} \, \hbar \omega
\]
so that
\[
E_{vac} = \sqrt{\frac{\hbar \omega}{2 \epsilon_0 \, V}}
\]
is the amplitude of the field in the dark vacuum, with $V$ being the volume of the cavity.\sidenote{This is the reason we require a cavity. Otherwise the integral would diverge.} One obtains the volume by integrating over full space, weighted by the local intensity:
\[
V =  \frac{1}{\text{max}(\boldsymbol{E}_c)^2} \, \int_\text{cavity} \boldsymbol{E}_c^2\, dr
\]
where $\boldsymbol{E}_c$ can be a field of any amplitude inside the cavity.

The electrical field of a single optical mode in a cavity then becomes\footcite[chap. 2.1 and 2.4]{GerryKnight2005}\footcite[chap. 6.1]{Rand2016}
\[
\hat{\boldsymbol{E}}(z,t) = \boldsymbol{x} \, E_{vac} \, (\hat{a} \, e^{i (k z - \omega t)} + \hat{a}^\dagger   \, e^{-i (k z - \omega t)} ) 
\]
where $\boldsymbol{x}$ is a unit vector defining the direction of polarization. 

\section{Pauli matrices for atoms}

The two-level system representing our atom is a spin $1/2$ system, i.e., a Fermion, not a Boson as the photons in the cavity. We can use operators similar to the ladder operators to excite ($  \hat{\sigma}_+$) or relax ($  \hat{\sigma}_-$) the two-level system
\[
 \hat{\sigma}_+ = \ket{e} \bra{g} \quad \text{and} \quad 
  \hat{\sigma}_- = \ket{g} \bra{e}
\]
The third operator to complete the Pauli spin algebra is the inversion operator, i.e. the third component of the Bloch vector
\[
 \hat{\sigma}_3 = \ket{e} \bra{e} - \ket{g} \bra{g}
\] 



\section{Jaynes-Cummings-Model}
Now we put everything together to  the Jaynes-Cummings-model. Sometimes this model is also called 'dressed atom' model.\footcite[chap. 6.8]{Rand2016} \footcite[chap. 4.5]{GerryKnight2005} \footcite[chap. 10.4]{Fox}  \footcite[chap. 3.4]{HarocheRaimond2006}

We construct an Hamiltonian of three parts:  atom, optical field, and light-matter interaction. The atom part is, using $\hbar \omega_0 = E_e - E_g$ 
\[
\hat{H}_A = \frac{1}{2} \, \hbar \omega_0 \, \hat{\sigma}_3
\]
where we have set the zero of the energy scale half way between ground and excited state. The optical field part is
\[
\hat{H}_F = \hbar \omega \, \hat{a} \hat{a}^\dagger
\]
with the optical frequency $\omega$ and the zero of the energy scale set to the vacuum energy. Light-matter interaction is given in the dipole approximation and neglecting terms that violate energy conservation by\footcite[chap. 6.7.1]{Rand2016}
\[
\hat{H}_I = - \boldsymbol{\mu} \, \boldsymbol{E} =
 \hbar g \, (\hat{\sigma}_+ \, \hat{a} + \hat{\sigma}_- \,\hat{a}^\dagger )
\]
Absorption of a photon ($\hat{a}$) excites the atom ($\hat{\sigma}_+$) and the other way round.
The coupling constant $g$ is given by
\[
 g = - \mu_{eg} \, \sqrt{\frac{\omega}{2 \hbar \, \epsilon_0 \, V}}
\]
where $\mu_{eg} $ is the projection of the transition dipole moment on the polarization direction of the light field. In total we have thus
\[
 \hat{H} = \frac{1}{2} \, \hbar \omega_0 \, \hat{\sigma}_3 
 +  \hbar \omega \, \hat{a} \hat{a}^\dagger
 + \hbar g \, (\hat{\sigma}_+ \, \hat{a} + \hat{\sigma}_- \,\hat{a}^\dagger )
\]

The idea of the Jaynes-Cummings-Model is to find eigen-states of this Hamiltonian. This is the same idea as in a coupled pendulum: the atom is one pendulum, the light field another, and the spring connecting the pendula is the coupling constant $g$. The uncoupled eigen-states are $\ket{g, n}$ and  $\ket{e, n-1}$, i.e. atom in ground or excited state, and either $n$ or $n-1$ photons in the cavity. For these two states, the Hamilton operator reads in matrix form
\[
\hat{H} = \hbar 
\begin{pmatrix}
n \omega - \frac{1}{2} \omega_0  & g \sqrt{n} \\
g \sqrt{n} & (n-1) \omega + \frac{1}{2} \omega_0 \\
\end{pmatrix}
\]
The new eigen-states are linear combinations of the old, obtained by diagonalizing the Hamilton operator in matrix form. For the eigen-energy we get
\[
E_\pm = \left( n - \frac{1}{2} \right) \hbar \omega \, \pm \, \frac{1}{2} \hbar
\sqrt{\Delta^2 + 4 |g|^2 n}
\]
where $\Delta = \omega_0 - \omega$
is energy difference between the uncoupled eigen-states, or the detuning between atom and field. The square-root is called generalized Rabi frequency $\Omega_R = \sqrt{\Delta^2 + 4 |g|^2 n}$. The new eigen-states are called dressed states $\ket{D_\pm}$ as the photons are 'dressing' the atom
\begin{eqnarray*}
\ket{D_+} = \sin \theta \ket{g,n} + \cos \theta \ket{e,n-1} \\
\ket{D_-} = \cos \theta \ket{g,n} - \sin \theta \ket{e,n-1} 
\end{eqnarray*}
with $\cos 2\theta = \Delta / \Omega_R$. On resonance, i.e. $\Delta = 0$, the two dressed states are the symmetric and anti-symmetric combinations of the uncoupled states.

\section{Mollow Triplet}

%https://demonstrations.wolfram.com/MollowTriplet/

%https://demonstrations.wolfram.com/CavityQuantumElectrodynamicsWithBosonsEmissionSpectraInTheSt/

\section{Vacuum Rabi Splitting}


\printbibliography[segment=\therefsegment,heading=subbibliography]
